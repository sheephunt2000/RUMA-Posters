\documentclass[12pt]{article}
\usepackage{amssymb,amsmath}
\usepackage{anyfontsize}
\thispagestyle{empty}
\usepackage{tikz}
\usetikzlibrary{calc}
\usepackage{setspace}
\usepackage{graphicx}
\usepackage[margin=0.65in]{geometry}
\usepackage{float}
\usepackage[letterspace=150]{microtype}


\begin{document}

%Border

\begin{tikzpicture}[overlay,remember picture]
\draw [line width=3pt,rounded corners=0pt,]
    ($ (current page.north west) + (1.25cm,-1.25cm) $)
    rectangle
    ($ (current page.south east) + (-1.25cm,1.25cm) $);       
\end{tikzpicture}
\begin{center}\includegraphics[scale=.35]{RUMAlogo.png}\\
presents... \\
\begin{spacing}{1}
{\fontsize{40}{44}\selectfont  \textsc{
Randomness in Computation}} \end{spacing}
 

~~\\
\begin{spacing}{1.5}
{\fontsize{24}{24} \selectfont A Lecture by Professor Shubhangi
  Saraf}  \end{spacing} 
\large Dept. of Mathematics, Rutgers University \\

\vspace{5mm}
\normalsize
\textsc{Abstract:}

\Large 
In this talk we will see the amazing power of randomness 
in computation. In a wide variety of settings such as cryptography, 
machine learning, sublinear time algorithms, and many many more,
the use of randomness has led to far more powerful and efficient algorithms 
than would have been possible otherwise.

\vspace{2mm}
I will also discuss some beautiful and quite surprising results
from complexity theory which indicate why people believe that for
efficient polynomial time computation, every randomized algorithm
can also be efficiently derandomized.

\vspace{5mm} 
\huge   \textsc{Monday\\October 14, 2019 \\Hill 423 at 6:00 pm}
\\

\vspace{5mm}
\large
\emph{*Pizza and refreshments will be served}
\end{center}
\begin{center}
  \large  Join our email list! Email us at math.ruma@gmail.com, or
go to bit.ly/RUMA2019.\\
\end{center}



\end{document}