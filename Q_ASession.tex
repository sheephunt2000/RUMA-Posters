
\documentclass[10pt]{article}
\usepackage{amsfonts,amsmath,amscd,amssymb,amsthm,mathrsfs,esint}
\usepackage{color,fancyhdr,framed,graphicx,verbatim,mathtools}

\usepackage[margin=1in]{geometry}

\usepackage{tikz}
\usetikzlibrary{matrix,arrows,decorations.pathmorphing}

\newtheorem{theorem}{Theorem}%[section]
\theoremstyle{definition}
\newtheorem{corollary}[theorem]{Corollary}
\newtheorem{lemma}[theorem]{Lemma}
\newtheorem{proposition}[theorem]{Proposition}
\newtheorem{answer}[theorem]{Answer}
\newtheorem{conjecture}[theorem]{Conjecture}
\newtheorem{definition}[theorem]{Definition}
\newtheorem{example}[theorem]{Example}
\newtheorem{exercise}[theorem]{Exercise}
\newtheorem{problem}[theorem]{Problem}
\newtheorem{question}[theorem]{Question}
\newtheorem{remark}[theorem]{Remark}
\newtheorem{result}[theorem]{Result}
\setcounter{theorem}{0}
\setcounter{section}{1}
\pagestyle{fancy}
\fancyhf{}
\rhead{Fall 2019}
\lhead{Q\&A Session - September, 9}
\rfoot{Page \thepage \hspace{1pt}}
\setcounter{secnumdepth}{-1} %section number display off

\renewcommand\qedsymbol{$\blacksquare$}

\begin{document}
\begin{center}
  {\bf Math Q\&A Session}
\end{center}
\begin{itemize}
  \item Question: Do you have a favorite math proof? (Adam)
  \item Answer: Adam - Applications of Burnside Lemma
  \item Question: I am a double major in math and physics (minor
    philosophy). Goal: math side of physics, but finding a hard
    time finding the intersection between math and physics for
    graduate school and PhD. How much disciplinary research is
    there in math and physics grad schools?
  \item Answer: Most people who do that route go into a PhD in
    physics, with the emphasis in theoretical physics (often). A
    small number of math programs will have a strong group of
    people in physics. At rutgers people in the math department
    also hold joint positions with physics department. So students
    usually enter the physics program for PhD. Brown has two math
    departments (math/applied math). What will be the most
    important is getting a feel about what kind of physics is most
    appealing to you, so you know the area of math as well. If
    you're thinking of a PhD program, it is nice to know what area
    of math you might be doing, but not necessary.
  \item Question: At what point in what course would you say
    really help determine what path you wanted to go in?
    I.e. theoretical, quant side, what classes do you want to go
    for?
  \item Answer: So I was considering pure math/theoretical cs. I
    did an REU, and got a mentor and did some research, and I felt
    that it was a really good experience and it really helped
  \item Answer: I really like the idea of doing something over the
    summer, another idea is to do an internship rather than an
    REU. Over the school years I was doing research in the EE
    department Over the summer I could try out stuff in
    industry. I didn't SWE, and when I tried finance, I also
    learned it wasn't for me. So, I'm currently looking for
    different branches. This helped me narrow down what I want to
    try out. I reccommend picking up a coding language that will
    make you appealing to companies. I.e. python. This will open
    up different
  \item Question: You said you didn't like your first internship?
    What didn't you like?
  \item Answer:
  \item Question: Talking about learning programming languages,
    there are many ways we can learn a language. It is very hard
    to measure how proficient one is at the language. What
    approach would you best recommend to display proficiency?
  \item Answer: There is a lot of ways, one of the best ways is
    to do a lot of side projects. I.e. create a personal website,
    want something to appeal to companies. Another way is to solve
    challenge programs, (RUCP, the club), as to show people what
    they're doing
  \item Answer: Also, codeforces and project euler. Rating of
    1500-1600 on codeforces is really good. Side projects are also
    good.
  \item Question: Can you explain how the honors track works?
  \item Answer: Leads to a B.S. in math science as it is credit
    intensive, and it makes sure that you have taken our two year
    long sequences 411, 412, 451, 452. Those courses will prepare
    you for graduate courses if you are going to do a graduate
    program in mathematics anywhere you go. How to get to that
    point or parts of that point before senior year? We usually
    invite students to honors sections like 291 and 251H, that are
    on our radar (Beals' radar). If you do well, Prof Beals will
    pay attention and check and see if you're ready for the next
    level. You would hope to do math 300 honors. We really expect
    students to appreciate a proof in the abstract, or construct a
    reasonable proof. If you've ogtten into math 300H, Prof. Beals
    will keep paying attention to you until you stumble. The
    students who do best in 300H, he will suggest 311H and 350H at
    the same time. If he is worried about it, he will recommend
    you do one of them in the time. Those courses will get you
    into 411,412 and 451, 452. Completing those will lead to a
    bachelors of science. That is the typical path, but you can
    also skip the lower 311H and 350H path by independently
    studying. That is not necsesarily recommended though. If you
    aren't in 350H and 311H, then you can still get into 451 and
    411 by doing well in the non-honors class. Another caveat is
    that the number of double majors is a lot (especially math
    majors), but that leads to time constraints. So a lot of our
    honors majors cannot take all 4. But 2 of them will lead to
    honors or high honors in math. If you can only take one, then
    you usually take the one that will best complement. The ones
    who have gotten through 411, 412, can take the serious
    graduate courses if they will still be at RU. But the standard
    math first year graduate courses are in analysis, algebra, and
    topology if given a strong enough preparation. Our strongest
    majors that have the advantage of focusing on math, will take
    a number of graduate courses. You have to have the good
    fortune of scheduling and focusing to take a lot of grad
    schools. It can be hard without being a sole math major
    though. I.e. math education majors have a hard time taking
    both of them. FOr all intensive purposes, you don't have to
    apply. 
  \item Question: The math department offers the 5 year BAMS
    program, for undergrads if we aren't too sure if we want the
    intenseive 4 year BS program, or the 5 year BAMS
    program, when will we have to decide?
  \item Answer: Look at the courses that intersect with
    undergraduate courses you want to take. Those who take 451,
    411, 412, 452 take a lot of math courses, that go with their
    interests. You sample around and talk to people as you develop
    interest, and talk to Beals about interests. Take courses at
    the undergraduate level that may lead to graduate courses
    you're interested in. At RU there is a lot of flexibility
    regarding courses and interests. Example: first year econ grad
    students like to take functional analysis. 
  \item Question: Can you speak about the research prospects for
    math related to computer science? (Graduate school) I.e. what
    is the benefit of having a math major on top of a cs major for
    cs grad school.
  \item Answer: It can depend on the cs
    interest. I.e. cryptopgrahy has heavy math. We have a lot of
    people who do comp sci and combinatorics. Comp sci and combo
    sort of mix into one big field.
  \item Answer: It is good to be good at math. There will be some
    aspect of proof in them with papers even if its just mostly
    empirical data. Having the math background to understand and
    drive to make those models better. Also math will help you
    differentiate you from others. Will make you look better on
    paper. 
  \item ANswer: Every cs major with a math background is stronger
    than a cs major. It is good to know math
  \item ANswer: Ask the CS department
  \item Question: Question for education: HOw did you choose to go
    into the educational track? I love to teach, but theres so
    many other things that it paralyzes me to look at all the
    other options. SO why did you want to go into education?
  \item Answer: coming into rutgers 2014, I know I wanted to go
    into math, one of my intersets in comp sci. I did tutoring at
    RU, and was hired to me a tutor as a sophomore. Tutoring these
    kids, they were mostly first generation students, so they
    seeked help for math and other things like how to deal with
    rutgers and financial aid. Things that are new to them and
    their family. In my sophomor eyear, my interests in education
    increased. I genuniely wanted to help these people. Looking at
    RU only having a 5 year program, I looked at it and thought it
    was doable for me to do the program at RU. The hardest thing
    about the teacher program is finishing the math degree on time
    with good grades. Based on experiences, that does seem to be
    the biggest obstacle, as you don't want to go past 4 years.
  \item Answer: If you like teaching, get a job teaching at RU!
    (LA, tutoring, etc.) Find an opportunity for teaching. Brian
    really liked teaching undergrads in his undergrad. Teaching at
    a higher level is nice. If Brian wants to teach fun math, then
    he ends up teaching at a university and not a middle
    school. 
  \item Answer: There is a lot of options at different
    levels. Don't discourage people who want to teach at a lower
    level. DOn't listen to those people. Don't model yourself
    about the worst aspects of people. If you go into pure math,
    unless you are going to IAS, you will be teaching. 
  \item Question: Besides education, comp sci, and finance, what
    are other options for people looking into a phd in pure math,
    but not commiting to academia forever.
  \item Answer: There are industry/tech positions and research at
    corporations. Also, there are different views of finance, so I
    would hesitate to group someone under 1 umbrella of
    finance. ALso economics is cool, and advertising as well. Also
    theres the NSA and research jobs, and nIST. Lots of
    interesting government jobs. Very much like an academic job,
    just no teaching. Lots of times doing the same abstract
    problems that you find interesting. 
  \item QUestion: Would you first get a phd?
  \item Answer: Don't need one to start, but you can. Do something
    you love.
  \item Question: HOw often do you see honors math students
    transition into law school.
  \item Answer: It happens, it happens. A stat I learned last year
    for the LSATS. Interesting trivia: The people with the highest
    LSATS are the math majors. 
  \item Question: For Brian - how is grad school life in general?
  \item ANswer: I really like it here. People talk about how grad
    school is a depression factory, this is true for a nontrivial
    number of people. But there are a lot of grad students who are
    happy with what they're doing. For those in math grad school,
    it is hard to accidently walk into math grad school. SO you
    want to make sure you like doing math, as well as teaching. If
    you don't like teaching very much, you may not be that
    happy. But I like it here: no homework, deal with undergrads,
    learn math. 
  \item Answer: The above may also be true for the theoretical
    sciences.
\end{itemize}
\end{document}
