\documentclass[12pt]{article}
\usepackage{amssymb,amsmath}
\usepackage{anyfontsize}
\thispagestyle{empty}
\usepackage{tikz}
\usetikzlibrary{calc}
\usepackage{setspace}
\usepackage{graphicx}
\usepackage[margin=0.65in]{geometry}
\usepackage{float}
\usepackage[letterspace=150]{microtype}


\begin{document}

%Border

\begin{tikzpicture}[overlay,remember picture]
\draw [line width=3pt,rounded corners=0pt,]
    ($ (current page.north west) + (1.25cm,-1.25cm) $)
    rectangle
    ($ (current page.south east) + (-1.25cm,1.25cm) $);       
\end{tikzpicture}
\begin{center}\includegraphics[scale=.45]{RUMAlogo.png}\\
presents... \\
\begin{spacing}{1}
{\fontsize{40}{44}\selectfont  \textsc{
When Euler met Morse (and maybe Floer)}} \end{spacing}
 

~~\\
\begin{spacing}{1.5}
{\fontsize{24}{24} \selectfont A Lecture by Professor Mariano
  Echeverria}  \end{spacing} 
\large Dept. of Mathematics, Rutgers University \\~~\\

\normalsize
\textsc{Abstract:}

\Large
A very useful way to classify surfaces in 3d space
is via their number of holes. There are many
way to compute such a number, but I will focus
on the Morse theory point of view, which involves studying
the behavior of a ``height'' function defined on the surface,
and a system of ODEs associated to it.

Time permitting, I will mention how this perspective can
be used to study more complicated spaces, which can even
be infinite dimensional! 
\\

\vspace{10mm} 
\Huge   \textsc{Tuesday\\November 12, 2019 \\Hill 423 at 7:00
  pm}

\vspace{2mm}
\large
\emph{*Pizza and refreshments will be served}
\end{center}

\begin{center}
  \large  Join our email list! Email us at math.ruma@gmail.com, or
go to bit.ly/RUMA2019.\\
\end{center}



\end{document}
