\documentclass[12pt]{article}
\usepackage{amssymb,amsmath}
\usepackage{anyfontsize}
\thispagestyle{empty}
\usepackage{tikz}
\usetikzlibrary{calc}
\usepackage{setspace}
\usepackage{graphicx}
\usepackage[margin=0.65in]{geometry}
\usepackage{float}
\usepackage[letterspace=150]{microtype}


\begin{document}

%Border

\begin{tikzpicture}[overlay,remember picture]
\draw [line width=3pt,rounded corners=0pt,]
    ($ (current page.north west) + (1.25cm,-1.25cm) $)
    rectangle
    ($ (current page.south east) + (-1.25cm,1.25cm) $);       
\end{tikzpicture}
\begin{center}\includegraphics[scale=.40]{RUMAlogo.png}\\
presents... \\
\begin{spacing}{1.5}
{\fontsize{36}{36}\selectfont  \textsc{
Algebraic structures and the algebraic structures on which they
live}} \end{spacing}
 

~~\\
\begin{spacing}{1.5}
{\fontsize{24}{24} \selectfont A Lecture by Professor Daniel Krashen}  \end{spacing}
\large Dept. of Mathematics, University of Pennsylvania \\~~\\
~~\\

\normalsize
\textsc{Abstract:}

\large
Mathematical objects often have two lives. On the one
hand, they are something like creatures living in complex
ecosystems, but on the other hand, they can also be like the
planets on which these creatures live. Sometimes these
``planetary'' algebraic structures can be given additional
geometric or topological features, giving a range of new tools
from other mathematical disciplines. 

In this talk I will try to push this metaphor, and use it as a
lens through which to view some open problems in the theory of
quadratic forms and division algebras. In particular, we will see
how these problems connect to ideas in geometry, such as the
(in)famous ``hearing the shape of the drum'' problem.
\\

\vspace{5mm} 
\huge   \textsc{Wednesday\\October 20, 2021 \\Hill 525 at 7:00
  pm}

\vspace{2mm}
\large
\emph{*Pizza and refreshments will be served at 6:30 at Hill 323}
\end{center}

\begin{center}
  \large  Join our email list! Email us at
  math.ruma@gmail.com\\Join our discord!
  https://discord.gg/vMvXqbS
\end{center}



\end{document}
