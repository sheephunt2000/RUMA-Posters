\documentclass[12pt]{article}
\usepackage{amssymb,amsmath}
\usepackage{anyfontsize}
\thispagestyle{empty}
\usepackage{tikz}
\usetikzlibrary{calc}
\usepackage{setspace}
\usepackage{graphicx}
\usepackage[margin=0.65in]{geometry}
\usepackage{float}
\usepackage[letterspace=150]{microtype}


\begin{document}

%Border

\begin{tikzpicture}[overlay,remember picture]
\draw [line width=3pt,rounded corners=0pt,]
    ($ (current page.north west) + (1.25cm,-1.25cm) $)
    rectangle
    ($ (current page.south east) + (-1.25cm,1.25cm) $);       
\end{tikzpicture}
\begin{center}\includegraphics[scale=.40]{RUMAlogo.png}\\
presents... \\
\begin{spacing}{1}
{\fontsize{40}{44}\selectfont  \textsc{
From Math to the Universe}} \end{spacing}
 

~~\\
\begin{spacing}{1.5}
{\fontsize{24}{24} \selectfont A Lecture by Professor Charles
  Keeton}  \end{spacing} 
\large Dept. of Physics, Rutgers University \\~~\\

\normalsize
\textsc{Abstract:}

\Large
In the 1930s Einstein used his theory of relativity to predict
that a star's gravity could act like a lens, bending light and
creating multiple images of a more distant star. Today this
``gravitational lensing'' is used across astrophysics and cosmology
to study galaxies, black holes, dark matter, and much more. The
phenomenon provides a beautiful connection between astrophysics
and mathematics. I will show examples of connections that involve
catastrophe theory, complex analysis, the Poincare-Hopf index
theorem, and stochastic processes. I will strive to explain how
astrophysical applications both motivate and benefit from deep
study of the mathematics of gravity and light. 
\\

\vspace{5mm} 
\Huge   \textsc{Wednesday\\March 4, 2020 \\Hill 525 at 6:00
  pm}

\vspace{2mm}
\large
\emph{*Pizza and refreshments will be served}
\end{center}



\end{document}
