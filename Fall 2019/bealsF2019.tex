\documentclass[12pt]{article}
\usepackage{amssymb,amsmath}
\usepackage{anyfontsize}
\thispagestyle{empty}
\usepackage{tikz}
\usetikzlibrary{calc}
\usepackage{setspace}
\usepackage{graphicx}
\usepackage[margin=0.65in]{geometry}
\usepackage{float}
\usepackage[letterspace=150]{microtype}


\begin{document}

%Border

\begin{tikzpicture}[overlay,remember picture]
\draw [line width=3pt,rounded corners=0pt,]
    ($ (current page.north west) + (1.25cm,-1.25cm) $)
    rectangle
    ($ (current page.south east) + (-1.25cm,1.25cm) $);       
\end{tikzpicture}
\begin{center}\includegraphics[scale=.50]{RUMAlogo.png}\\
presents... \\
\begin{spacing}{1}
{\fontsize{36}{36}\selectfont  \textsc{
What is Homotopy Theory?}} \end{spacing}
 

~~\\
\begin{spacing}{1.5}
{\fontsize{24}{24} \selectfont A Lecture by Professor Ian
  Coley}  \end{spacing} 
\large Dept. of Mathematics, Rutgers University \\~~\\
~~\\

\normalsize
\textsc{Abstract:}

\LARGE
Multivariable calculus naturally leads to the study of manifolds,
spaces that look like our first friend $\mathbb{R}^n$. But what if
we went 
the other way, where we stopped thinking about rigid spaces and
starting thinking more floppily? I'll introduce you to homotopy
theory: its origins, some cool little tools, and how it ties in
with both algebra and geometry.
\\

\vspace{5mm} 
\Huge   \textsc{Thursday\\February 3, 2022 \\Hill 525 at 7:30
  pm}

\vspace{2mm}
\large
\emph{*Pizza and refreshments will be served}
\end{center}

\begin{center}
  \Large  Join our email list! Email us at
  math.ruma@gmail.com\\Join our discord!
  https://discord.gg/vMvXqbS
\end{center}



\end{document}
