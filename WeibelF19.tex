\documentclass[12pt]{article}
\usepackage{amssymb,amsmath}
\usepackage{anyfontsize}
\thispagestyle{empty}
\usepackage{tikz}
\usetikzlibrary{calc}
\usepackage{setspace}
\usepackage{graphicx}
\usepackage[margin=0.65in]{geometry}
\usepackage{float}
\usepackage[letterspace=150]{microtype}


\begin{document}

%Border

\begin{tikzpicture}[overlay,remember picture]
\draw [line width=3pt,rounded corners=0pt,]
    ($ (current page.north west) + (1.25cm,-1.25cm) $)
    rectangle
    ($ (current page.south east) + (-1.25cm,1.25cm) $);       
\end{tikzpicture}
\begin{center}\includegraphics[scale=.40]{RUMAlogo.png}\\
presents... \\
\begin{spacing}{1}
{\fontsize{40}{44}\selectfont  \textsc{
Scissors and K-theory}} \end{spacing}
 

~~\\
\begin{spacing}{1.5}
{\fontsize{24}{24} \selectfont A Lecture by Professor Charles
  Weibel}  \end{spacing}
\large Dept. of Mathematics, Rutgers University \\~~\\
~~\\

\normalsize
\textsc{Abstract:}

\Large
If you draw any polygon on a sheet of paper, you can use scissors
to cut it up and rearrange the pieces to get a new polygon. These
two polygons are said to be ``scissors congruent.''  Concatenating
them gives an addition of polygons up to scissors congruence. We
can also subtract them if we use virtual polygons; this is the
basic principle of K-theory.  (Two polygons with the same area are
always scissors congruent!)

Now consider the problem in 3-space, with polyhedra instead
of polygons, and study congruence classes of polyhedra. (Ok, use a
knife.) Now the problem is to find an invariant other than volume.
\\

\vspace{10mm} 
\Huge   \textsc{Wednesday\\October 30, 2019 \\Hill 425 at 7:00
  pm}

\vspace{2mm}
\large
\emph{*Pizza and refreshments will be served}
\end{center}

\begin{center}
  \large  Join our email list! Email us at math.ruma@gmail.com, or
go to bit.ly/RUMA2019.\\
\end{center}



\end{document}
