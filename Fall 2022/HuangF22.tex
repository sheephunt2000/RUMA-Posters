\documentclass[12pt]{article}
\usepackage{amssymb,amsmath}
\usepackage{anyfontsize}
\thispagestyle{empty}
\usepackage{tikz}
\usetikzlibrary{calc}
\usepackage{setspace}
\usepackage{graphicx}
\usepackage[margin=0.65in]{geometry}
\usepackage{float}
\usepackage{hyperref}
\usepackage[letterspace=150]{microtype}
\usepackage{array}

\graphicspath{{../}} % looks in parent directory for image

\begin{document}

%Border

\begin{tikzpicture}[overlay,remember picture]
\draw [line width=3pt,rounded corners=0pt,]
    ($ (current page.north west) + (1.25cm,-1.25cm) $)
    rectangle
    ($ (current page.south east) + (-1.25cm,1.25cm) $);       
\end{tikzpicture}
\begin{center}\includegraphics[scale=.3]{RUMAlogo.png}\\
\large  presents... \\

\vspace{2mm}
\begin{spacing}{2}
{\fontsize{42}{28}\selectfont  \textsc{
    Representation theory of geometric objects and quantum theory
    }} \end{spacing}
 
~~\\
\begin{spacing}{1}
{\fontsize{24}{24} \selectfont A lecture by Professor XXX Huang}  \end{spacing} 
\large Dept. of Mathematics, Rutgers University \\~~\\

\normalsize
\textsc{Abstract:}

\Large 
Quantum mechanics and quantum field theory play a fundamental role in physics. Mathematically, they can be viewed as the representation theory of geometric objects, that is, the theory about how to study geometric objects using vector spaces and linear transformations. I will give an introduction to this mathematical approach to quantum theories.


\begin{spacing}{2}
    {\fontsize{36}{28}\selectfont  \textsc{
        Friday, October 28, 2022 \\Hill 525 at 6:00 pm
    }} 
\end{spacing}

\Large  Check out our website at \\
\url{https://ruma.rutgers.edu}.
\end{center}
\vspace{0.1mm}
\begin{center}
	\begin{tabular}{m{10em} m{1mm} m{11em}}
		\Large Join our Discord! & &
		\includegraphics[scale=0.07]{discord.png}
	\end{tabular}
\end{center}

\end{document}
