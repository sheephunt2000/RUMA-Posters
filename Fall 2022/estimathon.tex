\documentclass[12pt,letterpaper]{article}
\usepackage{amsmath,amsthm,amscd}
\usepackage{enumerate}
\usepackage{geometry}
\geometry{
	letterpaper,
	textwidth=7in,
	bottom=1.5in
}

\usepackage{titling}
\setlength{\droptitle}{-1.5in}

\title{Estimathon}
\author{An AWM/RUMA Event}
\date{November 2022}

\begin{document}

\maketitle

\section*{Overview}

Your team will have \textbf{30 minutes} to work on \textbf{13} estimation problems. The answer to each problem is a positive number. Your team will submit \textbf{intervals} for each problem. Intervals may not contain negative numbers or zero. You may not use the Internet, a calculator, or any other type of external reference material to solve these problems.

\section*{Scoring}

An interval is \textbf{good} if it contains the correct answer. After the 30 minutes is over, the final score for your team will be:

\[\left(10+\sum_{\text{good intervals}}\left\lfloor\frac{\text{max}}{\text{min}}\right\rfloor\right)\cdot 2^{13-(\text{\# of good intervals})}\]

That is, for every problem you get wrong (or leave blank), your score doubles, and you are incentivized to minimize the intervals you submit. The winning team is the team with the \textbf{LOWEST SCORE}.

\section*{Submitting intervals}

Every team can submit up to \textbf{18 total intervals}. Your team will receive an answer sheet containing 18 slips. Use these to submit your intervals \textbf{at any time} throughout the contest. Each slip must contain your \textbf{team name}, \textbf{problem number}, and \textbf{interval} (min and max value). You can bring slips up at any time during the 30 minutes; we will attempt to grade entries in real time.

\section*{Resubmitting}

Since you have up to 18 submissions for 13 problems, you may submit intervals for a given problem more than once. Only the last submission for any given problem is the one that will count towards your final score.

\section*{Notation}

You may use scientific notation if you like, but nothing more complicated than that. For instance, the interval \([3\cdot 10^6,10^7]\) is fine, but \([3^7,4^8]\) is not. 

\newpage

\section*{Problems}

\begin{enumerate}
	\itemsep 1.25em 
	\item How many daily riders does the Rutgers--New Brunswick bus system have?
	\item How many rivets are in the Eiffel Tower?
	\item How many calories would you consume if you had one of everything off of the Taco Bell menu? (This includes drinks!)
	\item What was the population of Middlesex County (where New Brunswick is) according to the 1960 census?
	\item How many people got a 5 (the highest score) on the 2019 AP Calculus BC exam?
	\item What is the melting point of iron in Kelvin?
	\item How many undergraduate majors does UCLA have?
	\item How much did The Sound of Music make in the box office (in 2021 USD)?
	\item How many Instagram followers does Taylor Swift have (as of 11/15/2022)?
	\item How many words are there in the Beatles album \textit{Sgt. Pepper's Lonely Hearts Club Band}?
	\item How much vitamin E is in a trillion croissants (in kg or lb)?
	\item How many USD\$1 bills were printed in fiscal year 2020?
	\item What is the area of the quadrilateral formed by the four Rutgers--New Brunswick dining halls in ft\(^2\) or m\(^2\)? (Busch, Livingston, Nielson, and Brower)
\end{enumerate}

\end{document}