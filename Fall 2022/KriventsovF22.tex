\documentclass[12pt]{article}
\usepackage{amssymb,amsmath}
\usepackage{anyfontsize}
\thispagestyle{empty}
\usepackage{tikz}
\usetikzlibrary{calc}
\usepackage{setspace}
\usepackage{graphicx}
\usepackage[margin=0.65in]{geometry}
\usepackage{float}
\usepackage{hyperref}
\usepackage[letterspace=150]{microtype}
\usepackage{array}

\graphicspath{{../}} % looks in parent directory for image

\begin{document}

%Border

\begin{tikzpicture}[overlay,remember picture]
\draw [line width=3pt,rounded corners=0pt,]
    ($ (current page.north west) + (1.25cm,-1.25cm) $)
    rectangle
    ($ (current page.south east) + (-1.25cm,1.25cm) $);       
\end{tikzpicture}
\begin{center}\includegraphics[scale=.3]{RUMAlogo.png}\\
\large  presents... \\

\vspace{2mm}
\begin{spacing}{1}
{\fontsize{26}{28}\selectfont  \textsc{
From optimization to free boundaries}} \end{spacing}
 
~~\\
\begin{spacing}{1.5}
{\fontsize{24}{24} \selectfont A lecture by Professor Dennis Kriventsov}  \end{spacing} 
\large Dept. of Mathematics, Rutgers University \\~~\\

\normalsize
\textsc{Abstract:}

\Large 
Calculus of variations is an area of analysis concerned with minimizing energies, generally over large collections of objects like functions or sets. I will describe how with some ideas from Calculus I (and maybe some hidden machinery) you can successfully find minimizers to such energies and then try to understand what they look like. Depending on the energy, this leads directly to partial differential equations and to free boundary problems (which are like partial differential equations satisfied by set boundaries). These subjects contain the most successful analysis developments of the past century--but also some basic questions which we just do not know how to answer.\\

\vspace{2.5mm} 
\huge   \textsc{Tuesday, September 27, 2022 \\Hill 525 at 6:00 pm} \\
\vspace{2.5mm}

\Large  Check out our website at \\
\url{https://ruma.rutgers.edu}.
\end{center}
\vspace{0.1mm}
\begin{center}
	\begin{tabular}{m{10em} m{1mm} m{11em}}
		\Large Join our Discord! & &
		\includegraphics[scale=0.07]{discord.png}
	\end{tabular}
\end{center}

\end{document}
