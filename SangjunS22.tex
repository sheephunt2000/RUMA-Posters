\documentclass[12pt]{article}
\usepackage{amssymb,amsmath}
\usepackage{anyfontsize}
\thispagestyle{empty}
\usepackage{tikz}
\usetikzlibrary{calc}
\usepackage{setspace}
\usepackage{graphicx}
\usepackage[margin=0.65in]{geometry}
\usepackage{float}
\usepackage[letterspace=150]{microtype}


\begin{document}

%Border

\begin{tikzpicture}[overlay,remember picture]
\draw [line width=3pt,rounded corners=0pt,]
    ($ (current page.north west) + (1.25cm,-1.25cm) $)
    rectangle
    ($ (current page.south east) + (-1.25cm,1.25cm) $);       
\end{tikzpicture}
\begin{center}\includegraphics[scale=.45]{RUMAlogo.png}\\
presents... \\
\begin{spacing}{1}
{\fontsize{34}{34}\selectfont  \textsc{A topological proof of Euclid's Theorem}} \end{spacing}

 

~~\\
\begin{spacing}{1.5}
{\fontsize{23}{23} \selectfont A Lecture by Sangjun Ko}  \end{spacing} 
\large Dept. of Mathematics, Rutgers University \\~~\\

\normalsize
\textsc{Abstract:}

\Large
Euclid's Theorem, which states that there are infinitely many primes, is a classic theorem whose proof is studied by every student who is beginning to learn about proofs. As an undergraduate, 2020 Abel Prize winner Hillel Furstenberg came up with a different proof which takes ideas from point-set topology. We will give a quick introduction to basic point-set topology and present Furstenberg's proof, and, if time permits, discuss some properties of the so-called ``evenly spaced integer topology" as well as other similar constructions.


\vspace{2mm} 
\huge   \textsc{Wednesday\\March 23, 2022 \\Hill 705 at 7:30
  pm}

\vspace{2mm}
\large
\emph{*Pizza and refreshments will be served}

  \Large  Join our email list! Email us at
  math.ruma@gmail.com\\Join our discord!
  https://discord.gg/vMvXqbS
\end{center}



\end{document}
