\documentclass[12pt]{article}
\usepackage{amssymb,amsmath}
\usepackage{anyfontsize}
\thispagestyle{empty}
\usepackage{tikz}
\usetikzlibrary{calc}
\usepackage{setspace}
\usepackage{graphicx}
\usepackage[margin=0.65in]{geometry}
\usepackage{float}
\usepackage[letterspace=150]{microtype}

\graphicspath{{../}} % looks in parent directory for image

\begin{document}

%Border

\begin{tikzpicture}[overlay,remember picture]
\draw [line width=3pt,rounded corners=0pt,]
    ($ (current page.north west) + (1.25cm,-1.25cm) $)
    rectangle
    ($ (current page.south east) + (-1.25cm,1.25cm) $);       
\end{tikzpicture}
\begin{center}\includegraphics[scale=.45]{RUMAlogo.png}\\
presents... \\
\begin{spacing}{1}
{\fontsize{33}{33}\selectfont  \textsc{Holomorphic Nullstellensatz}} \end{spacing}

 

~~\\
\begin{spacing}{1.5}
{\fontsize{23}{23} \selectfont A Lecture by Colin Fan}  \end{spacing} 
\large Dept. of Mathematics, Rutgers University \\~~\\

\normalsize
\textsc{Abstract:}

\large
Hilbert proved his Nullstellensatz (German for ``theorem of zeros'') in 1893, establishing a duality (1-1 correspondence) between geometry and algebra in the setting of polynomials. In particular, the Nullstellensatz allows us to use algebra (easier) to study geometry (harder). 

Being more ambitious, we can ask whether or not we can establish a duality between geometry and algebra in the setting of holomorphic functions. This question is intrinsically interesting, and is further motivated by the need to study complex manifolds. Thankfully, Rückert, in 1931 established his analytic Nullstellensatz, being the first one to introduce algebraic tools to complex analytic geometry. 

In this talk, I hope to keep the prerequisites at a minimum with little to no proofs and highlight the beauty of both the Hilbert and the Rückert Nullstellensatz, as well as the difficulties one may face in the study of holomorphic functions of several variables. 

\vspace{2mm} 
\huge   \textsc{Wednesday\\April 20, 2022 \\Hill 525 at 7
  pm}

\vspace{2mm}
\large
\emph{*Pizza and refreshments will be served}

  \Large  Join our email list! Email us at
  math.ruma@gmail.com\\Join our discord!
  https://discord.gg/vMvXqbS
\end{center}



\end{document}
