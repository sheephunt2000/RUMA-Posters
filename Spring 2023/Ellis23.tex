\documentclass[12pt]{article}
\usepackage{amssymb,amsmath}
\usepackage{anyfontsize}
\thispagestyle{empty}
\usepackage{tikz}
\usetikzlibrary{calc}
\usepackage{setspace}
\usepackage{graphicx}
\usepackage[margin=0.65in]{geometry}
\usepackage{float}
\usepackage{hyperref}
\usepackage[letterspace=150]{microtype}
\usepackage{array}

\urlstyle{rm}

\graphicspath{{../}} % looks in parent directory for image

\begin{document}

%Border

\begin{tikzpicture}[overlay,remember picture]
\draw [line width=3pt,rounded corners=0pt,]
    ($ (current page.north west) + (1.25cm,-1.25cm) $)
    rectangle
    ($ (current page.south east) + (-1.25cm,1.25cm) $);       
\end{tikzpicture}
\begin{center}\includegraphics[scale=.4]{RUMAlogo.png}\\
\large  presents... \\

\vspace{1mm}
\begin{spacing}{2}
{\fontsize{32}{18}\selectfont  \textsc{
    The Four Numbers Game
    }} \end{spacing}
 
\begin{spacing}{1}
{\fontsize{24}{18} \selectfont A lecture by Professor Paul Ellis}  \end{spacing} 
\large Dept. of Mathematics, Rutgers University \\~~\\

\normalsize

\vspace{10mm}

\textsc{Abstract:}



\LARGE
Write any four numbers at the corners of a square.  Next, write the difference of each pair at the midpoint of the corresponding side.  Connect these midpoints to make a smaller square inscribed in the original one.  Repeat the process.  What happens?  Always?  Can you prove it?

\vspace{10mm}

\begin{spacing}{1.5}
    {\fontsize{24}{28}\selectfont  \textsc{
        Friday, Match 9th, 2022 \\ Hill 705 at 7:30 PM}
    } 
\end{spacing}

\Large  Check out our website at 
\url{https://ruma.rutgers.edu}!
\end{center}
\vspace{0.1mm}
\begin{center}
	\begin{tabular}{m{15em} m{1mm} m{5em}}
		\Large {\textbf{Join our Discord!}} & &
		\includegraphics[scale=0.07]{discord.png}
	\end{tabular}
\end{center}


\end{document}
