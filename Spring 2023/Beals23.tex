\documentclass[12pt]{article}
\usepackage{amssymb,amsmath}
\usepackage{anyfontsize}
\thispagestyle{empty}
\usepackage{tikz}
\usetikzlibrary{calc}
\usepackage{setspace}
\usepackage{graphicx}
\usepackage[margin=0.65in]{geometry}
\usepackage{float}
\usepackage{hyperref}
\usepackage[letterspace=150]{microtype}
\usepackage{array}

\urlstyle{rm}

\graphicspath{{../}} % looks in parent directory for image

\begin{document}

%Border

\begin{tikzpicture}[overlay,remember picture]
\draw [line width=3pt,rounded corners=0pt,]
    ($ (current page.north west) + (1.25cm,-1.25cm) $)
    rectangle
    ($ (current page.south east) + (-1.25cm,1.25cm) $);       
\end{tikzpicture}
\begin{center}\includegraphics[scale=.2]{RUMAlogo.png}\\
\large  presents... \\

\vspace{1mm}
\begin{spacing}{2}
{\fontsize{28}{18}\selectfont  \textsc{
    Generalized Functions and Generalized Derivatives
    }} \end{spacing}
 
\begin{spacing}{1}
{\fontsize{18}{18} \selectfont A lecture by Professor Michael Beals}  \end{spacing} 
\large Dept. of Mathematics, Rutgers University \\~~\\

\normalsize
\textsc{Abstract:}

\LARGE
How do we make sense of the derivative of a discontinuous function?  How do we make sense of the sum of a non-converging series of functions?  How do we take half of a derivative of a function?  And why would we want to do any of these things?  The answer to that last question is, perhaps, because the physicists and mathematicians of the nineteenth century needed the results.  And the resulting theory of generalized functions changed analysis in the twentieth century.  We will consider how it all works.


\begin{spacing}{1.5}
    {\fontsize{24}{28}\selectfont  \textsc{
        Friday, February 3rd, 2022 \\ Hill XXXX at 5:00 PM}
    } 
\end{spacing}

\Large  Check out our website at 
\url{https://ruma.rutgers.edu}!
\end{center}
\vspace{0.1mm}
\begin{center}
	\begin{tabular}{m{15em} m{1mm} m{5em}}
		\Large {\textbf{Join our Discord!}} & &
		\includegraphics[scale=0.07]{discord.png}
	\end{tabular}
\end{center}


\end{document}
